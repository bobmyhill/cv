\documentclass{scrartcl}	% classe article di KOMA
\reversemarginpar
\newcommand{\MarginDate}[1]{\marginpar{\raggedleft\itshape\small#1}\\}
\newcommand{\MarginDateb}[1]{\marginpar{\raggedleft\itshape\small#1}}
%\usepackage[latin1]{inputenc}	% la codifica di input
%\usepackage[T1]{fontenc}	% la codifica dei font
\usepackage[LabelsAligned]{currvita}	% un buon pacchetto per CV
\usepackage[nochapters]{classicthesis} % stile ClassicThesis
\usepackage[a4paper, footskip=0.5cm, top=1.5cm, bottom=1.5cm, left=4cm, right=3cm]{geometry}
\usepackage{url}	% per gli indirizzi Internet
\renewcommand{\cvheadingfont}{\LARGE\color{Maroon}}
\renewcommand{\cvlistheadingfont}{\large}
\renewcommand{\cvlabelfont}{\qquad}
\setlength{\parindent}{-0.5cm}
% PDF SETUP
\usepackage{multicol}
\usepackage{color} 
\definecolor{myblue}{rgb}{0,0.2,0.4}
\date{}
% Setup hyperref package, and colours for links, text and headings
% ---- FILL IN HERE THE DOC TITLE AND AUTHOR
\usepackage{hyperref} 		
\hypersetup{	
  colorlinks,
  breaklinks,
  urlcolor=myblue, 
  linkcolor=myblue,
  pdftitle={Robert Myhill - Curriculum Vitae},
  pdfauthor={Robert Myhill}}
\usepackage{eurosym}
\usepackage{setspace}

\newlength{\datebox}\settowidth{\datebox}{\today}

\newenvironment{changemargin}[2]{%
  \begin{list}{}{%
    \setlength{\topsep}{0pt}%
    \setlength{\leftmargin}{#1}%
    \setlength{\rightmargin}{#2}%
    \setlength{\listparindent}{\parindent}%
    \setlength{\itemindent}{\parindent}%
    \setlength{\parsep}{\parskip}%
  }%
    \item[]
  \end{list}}

\newcommand{\NewWorkExperience}[3]{\noindent\hangindent=2em\hangafter=0 \parbox{\datebox}{\textit{#1}}\hspace{1.5em} #2 #3%
\vspace{0.5em}\\}
\newcommand{\Contact}[3]{\noindent\hangindent=0em\hangafter=0 \parbox{\datebox}{\textit{#1}}\hspace{1.5em} #2 #3\\}
\newcommand{\SubHeading}[2]{\noindent\hangindent=2em\hangafter=0 #1 #2%
\vspace{0.5em}\\}
\newcommand{\Description}[1]{\noindent\hangindent=2em\hangafter=0\footnotesize{#1}\par\normalsize}
\newcommand{\Heading}[1]{\hangindent=2em\hangafter=0\hspace{-0.25cm}\spacedlowsmallcaps{#1}\vspace{0.5em}\\}
\newcommand{\Sep}{\hspace{1cm} \\}
\begin{document}
\thispagestyle{empty}
\begin{cv}

\begin{center}
\setlength{\leftmargin}{2.5cm}
\textsc{Curriculum Vitae}
\vspace{8pt}
\hrule
\vspace{8pt}
\LARGE \textsc{Robert Myhill}
\end{center}
\vspace{8pt}
\noindent Department of Earth Sciences \\
University of Cambridge \\
Bullard Laboratories \\
Madingley Rise \\
Cambridge, CB3 0EZ \\
UK \\
\hspace{1.5em}\\
\Contact{phone}{+44 (0)7841 714164}{}
\Contact{fax}{+44 (0)1223 360779}{}
\Contact{email}{\href{mailto:rm438@cam.ac.uk}{rm438@cam.ac.uk}}{}
%%\NewWorkExperience{\textsc{url}}{\href{http://www.esc.cam.ac.uk/people/research-students/bob-myhill}{http://www.esc.cam.ac.uk/people/research-students/bob-myhill}}\\ 


\vspace{1.0em}
\Heading{Education}
\SubHeading{PhD \MarginDateb{Current}Research at}{the University of Cambridge\vspace{0.5em}\\\textit{The Mechanisms of Deep-Focus Earthquakes}}
\Description{
I am researching the causes of earthquakes occuring at depths greater than 300 km through an integrated analysis of seismic distributions, focal mechanisms and the pressure-temperature regimes within which these events occur. I will combine this analysis with geophysical modelling techniques to elucidate aspects of the interaction between subducting plates and the upper-lower mantle boundary.\\
\emph{Advisors: Dan McKenzie and Keith Priestley}.
}
\Sep
\Description{
\textsc{MSci + BA} \MarginDateb{2004-2008} Peterhouse, University of Cambridge, Class of 2008. \\
Natural Sciences (Physical; 4 years).\\
Part III: First Class. 1/36 in Class (Geological Sciences).\\
Part II: First Class. 1/39 in Class (Geological Sciences).\\
Part IB: First Class (Maths, Stratigraphic Geology, Mineralogy, Petrology).\\
Part IA: First Class (Geology, Maths, Physics, Chemistry).
}



\vspace{1.0em}
\Heading{Selected grants and awards}
\Description{
The Kingsley Bye-Fellowship\MarginDateb{2010}, Magdalene College Cambridge.\\
The Hugo\MarginDateb{2008} de Balsham Prize for Exceptional Academic Distinction.\\
The Harkness Scholarship (first-placed Finalist in Geological Sciences, University of Cambridge).\\
The Huppert Prize in Geophysics.\\
The Henry Wilkinson Cookson Senior Scholarship \MarginDateb{2007} in Natural Sciences.\\
The John Reekie Memorial Prize for the best geological fieldwork-based thesis submitted for the first degree at the Department of Earth Sciences.\\
Peterhouse Junior and Senior Scholarships\MarginDateb{2005-2006}.\\
Departmental Field Mapping Prizes \MarginDateb{2005-2007}\\ (5 prizes, from the Arran, Sedbergh, Dorset and Cornwall and Greece Field Trips).\\
The George Watson Prize for Outstanding Scholarship \MarginDateb{2004}\\
(Best results, Class of 2004, Sir John Leman High School).\\
St. John Ambulance Grand Prior Award. \MarginDate{2003}
}
\vspace{1.0em}
\Heading{Skills}
\Description{
\vspace{-1.5em}
\begin{itemize}
\item Competent user of \LaTeX, Microsoft and Serif Office programs.
\item Experience of FORTRAN, C, C++ languages, and use of the OpenGL API.
\item Experience in BASH and HTML scripting.
\item Over 400 hours experience with THERMOCALC.
\end{itemize}
}

\clearpage

\vspace{1.0em}
\Heading{Publications and presentations}
\SubHeading{Peer-reviewed journal articles}{}
\Description{
R. Myhill, D. McKenzie, K. Priestley, \MarginDateb{2010} ``Clustering of deep-focus earthquakes in the southwest Pacific''. \\
\emph{Submitted to Earth and Planetary Science Letters.}\\
R. Myhill, \MarginDateb{2010} ``Constraints on evolution of the Mesohellenic Ophiolite from sub-ophiolitic metamorphic rocks''. \\
\emph{Geological Society of America Special Publication, accepted.}\\
A. Rassios, Y. Dilek, R. Myhill, D. Ghikas, A. Mpatsi, \MarginDateb{2010} ``Melange Formations beneath the Pindos Basin Ophiolites, Northern Greece: Evidence of an “active,” rapid decollement emplacement surface''. \\
\emph{Geological Society of America Special Publication, accepted.}
}

\Sep
\SubHeading{Presentations}{}
\Description{
``The search for structure: deep-focus earthquakes''. \MarginDateb{2010} \emph{Cambridge Earth Sciences Graduate Talk.}\\
``Clustering of deep-focus \MarginDateb{2009} earthquakes in the southwest Pacific''. \emph{Poster presentation, American Geophysical Union Fall Meeting.}\\
``Faulting 300 kilometers down: \MarginDateb{2009} The mystery of deep-focus earthquakes''. \emph{Magdalene Parlour Talk.}\\
``Deep-focus earthquakes''. \emph{Bullard Laboratories Friday Talk.}\\
``The significance of high temperature \MarginDateb{2008} low pressure rocks beneath the Mesohellenic Ophiolite''. \emph{Poster presentation, Institute of Geology and Mineral Exploration (IGME) Field Symposium.}\\
IGME Field Symposium: Ophiolites 2008. \emph{Field demonstrator and member of the organising team.}\\
K\"{o}ln undergraduate field trip to Greece. \emph{1-day invited field demonstrator.}
}


\vspace{1.0em}
\Heading{Relevant Experience}
\Description{
Masters Thesis at the University of Cambridge: \MarginDateb{06/07-01/08} Metamorphic Development beneath the Mesohellenic Ophiolite. I obtained a high First (80\%) for this project.\\
\emph{Advisor: Dr. Timothy Holland, University of Cambridge.}
}
\Sep
\Description{
Bachelors Thesis at the University of Cambridge. \MarginDateb{06/06-01/07} Independent mapping project and industrial work experience: Vourinos, Northern Greece. I obtained the top First in the year for this project (80\%).\\
\emph{Advisors: Dr. Alan Smith, University of Cambridge and Dr. Anne Rassios, IGME.}
}
\Sep
\Description{
Field geologist, British Geological Survey. \MarginDateb{07/05-08/05} Paid appointment for the Tellus Project, part of a national environmental survey completed in 2006.\\
\emph{Advisors: Louise Ander, Sean Quigley, Sophia Passmore (British Geological Survey).}
}
\Sep
\Description{
Supervisions \MarginDateb{2008-present}given in the following courses: \emph{Geology (IA), Hydrosphere (IB), Tectonics and Structural Geology (IB), Tectonics (II/III) and Revision and Essay Skills (IA/IB/II/III).}\\
Demonstrated practicals for the following courses: \emph{Tectonics and Structural Geology (IB), Tectonics (II/III) and Seismology (II/III).}\\
Field Demonstrator: \emph{Ketton (IA), Arran (IA), Sedbergh (IB)}.
}
\Sep
\Description{
250+ \MarginDateb{2005-present}days fieldwork experience (as of \today) as demonstrator, field guide, employee, researcher and student.\\
\emph{(Locations include: Ireland, Iceland, Greece, Dorset, Cornwall, Sedbergh, and the Isles of Arran and Skye.)}
}

\vspace{1.0em}
\Heading{Other interests}
\Description{
\vspace{-1.5em}
\begin{itemize}
\item First Aid: I am presently Duties Coordinator for Cambridge LINKS, the student division of St John Ambulance, having been a member since 1998.
\item Magdalene College MCR: I am currently MCR Treasurer.
\item Outreach: I volunteer for Time Truck (the student geological outreach organisation) and also SEEK (Science and Engineering Experiments for Kids).
\item The Sedgwick Club (The University of Cambridge geological society): As an undergraduate I was president of this society, and I remain an active member.
\item Foreign travel: I particularly enjoy my time in Greece, and spent my final year at undergraduate level learning Modern Greek at the Department of Medieval and Modern Languages.
\end{itemize}
}

\enlargethispage{\baselineskip}
\end{cv}
\end{document}
