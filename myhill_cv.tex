%% start of file `template.tex'.
%% Copyright 2006-2012 Xavier Danaux (xdanaux@gmail.com).
%
% This work may be distributed and/or modified under the
% conditions of the LaTeX Project Public License version 1.3c,
% available at http://www.latex-project.org/lppl/.


\documentclass[10pt,a4paper,sans]{templates/moderncv}   % possible options include font size ('10pt', '11pt' and '12pt'), paper size ('a4paper', 'letterpaper', 'a5paper', 'legalpaper', 'executivepaper' and 'landscape') and font family ('sans' and 'roman')

% moderncv themes
\moderncvstyle{classic}                        % style options are 'casual' (default), 'classic', 'oldstyle' and 'banking'
\moderncvcolor{myhill}                          % color options 'blue' (default), 'orange', 'green', 'red', 'purple', 'grey' and 'black'
%\renewcommand{\familydefault}{\sfdefault}    % to set the default font; use '\sfdefault' for the default sans serif font, '\rmdefault' for the default roman one, or any tex font name
%\nopagenumbers{}                             % uncomment to suppress automatic page numbering for CVs longer than one page

% character encoding
%\usepackage[utf8]{inputenc}                  % if you are not using xelatex ou lualatex, replace by the encoding you are using
%\usepackage{CJKutf8}                         % if you need to use CJK to typeset your resume in Chinese, Japanese or Korean

% adjust the page margins
\usepackage[scale=0.80, top=1.5cm, bottom=1.0cm]{geometry}
%\setlength{\hintscolumnwidth}{3cm}           % if you want to change the width of the column with the dates
%\setlength{\maketitlenamewidth}{10cm}        % for the 'classic' style, if you want to force the width allocated to your name and avoid line breaks. be careful though, the length is normally calculated to avoid any overlap with your personal info; use this at your own typographical risks...

\usepackage{enumitem}

\newcommand{\doi}[1]{\href{http://dx.doi.org/#1}{doi:#1}}

% personal data
\firstname{Robert}
\familyname{Myhill}
\title{Postdoctoral Fellow, BGI}      % optional, remove the line if not wanted
\address{Bayerisches Geoinstitut}{Universit\"at Bayreuth}{95440 Bayreuth, Deutschland}    % optional, remove the line if not wanted
\mobile{+44-(0)~7841~61~4164}                  % optional, remove the line if not wanted
\phone{+49-(0)~921~55~3749}
\fax{+49-(0)~921~55~3769}                      % optional, remove the line if not wanted
\email{bob.myhill@uni-bayreuth.de}                          % optional, remove the line if not wanted
\photo[64pt][0.4pt]{photo/photo}                  % '64pt' is the height the picture must be resized to, 0.4pt is the thickness of the frame around it (put it to 0pt for no frame) and 'picture' is the name of the picture file; optional, remove the line if not wanted

% to show numerical labels in the bibliography (default is to show no labels); only useful if you make citations in your resume
%\makeatletter
%\renewcommand*{\bibliographyitemlabel}{\@biblabel{\arabic{enumiv}}}
%\makeatother

% bibliography with mutiple entries
%\usepackage{multibib}
%\newcites{book,misc}{{Books},{Others}}
%----------------------------------------------------------------------------------
%            content
%----------------------------------------------------------------------------------
\begin{document}
%\begin{CJK*}{UTF8}{gbsn}                     % to typeset your resume in Chinese using CJK
%-----       resume       ---------------------------------------------------------
\makecvtitle

\section{Personal Information}
\cvitem{Full Name}{Robert Christopher Myhill}
\cvitem{Date of birth}{16 March 1986}
\cvitem{Place of Birth}{Gorleston, Norfolk, UK}
\cvitem{Nationality}{British}
%\cvitem{Address}{Emil-Warburg-Weg 36, 95447 Bayreuth, Germany} 

\section{Employment}
\cventry{2015}{Postdoctoral Researcher}{Bayerisches Geoinstitut}{University of Bayreuth}{}{Project: Thermodynamics of light element incorporation into metallic melts and differentiation of the terrestrial planets}
\cventry{2013-2014}{Humboldt Postdoctoral Fellow}{Bayerisches Geoinstitut}{University of Bayreuth}{}{Project: Phase compositions and melting relations of hydrous and ferric-rich phases}
\cventry{2012}{Visiting Scientist}{Bayerisches Geoinstitut}{University of Bayreuth}{}{Project: High pressure volatile-rich melting}

\section{Education}
\cventry{2008--2012}{PhD in Earth Sciences}{Magdalene College}{University of Cambridge}{}{\textbf{The Mechanisms of Deep Earthquakes}\\ In this thesis, I conducted a study of earthquakes occurring at depths greater than 60 km through an integrated analysis of seismic distributions, focal mechanisms and conditions within subducting slabs. \emph{Advisors: Professors Dan McKenzie and Keith Priestley}.}  % arguments 3 to 6 can be left empty
\cventry{2004--2008}{MSci + BA}{Peterhouse}{University of Cambridge}{\textit{First Class}}{\textbf{Natural Sciences (Physical; 4 years})
\begin{itemize}
\item Part III: First Class. 1/36 in Class (Geological Sciences).
\item Part II: First Class. 1/39 in Class (Geological Sciences).
\item Part IB: First Class (Maths, Stratigraphic Geology, Mineralogy, Petrology).
\item Part IA: First Class (Geology, Maths, Physics, Chemistry).
\end{itemize}}

\section{Selected Grants and Awards}
\cvitem{2012}{Alexander von Humboldt Research Fellowship for Postdoctoral Researchers (2013-2014).}
\cvitem{2011}{Outstanding Student Poster Award, Geodynamics Division (European Geophysical Union General Assembly).}
\cvitem{2010}{The Kingsley Bye-Fellowship. Magdalene College, Cambridge.}
\cvitem{2008}{The Hugo de Balsham Prize for Exceptional Academic Distinction. Peterhouse, Cambridge.}
\cvitem{}{The Harkness Scholarship (first-placed Finalist in Geological Sciences, University of Cambridge).}
%\cvitem{}{The Huppert Prize in Geophysics}{}
\cvitem{2007}{The Henry Wilkinson Cookson Senior Scholarship in Natural Sciences, University of Cambridge.}
\cvitem{}{The John Reekie Memorial Prize for the best geological fieldwork-based thesis submitted for the first degree at the Department of Earth Sciences, University of Cambridge.}
%\cvitem{2005-2006}{Peterhouse Junior and Senior Scholarships.}{}
%\cvitem{2005-2007}{Departmental Field Mapping Prizes}{(5 prizes, from the Arran, Sedbergh, Dorset and Cornwall and Greece Field Trips).}
\cvitem{2004}{The George Watson Prize for Outstanding Scholarship (Best results, Class of 2004, Sir John Leman High School).}
%\cvitem{2003}{St John Ambulance Grand Prior Award.}{}

\clearpage

\section{Publications and presentations}
\subsection{Peer-reviewed journal articles}
%\cvitem{2015}{\tiny{Oxygen fugacity and melting in the deep upper mantle}}
%\cvitem{2015}{\tiny{Partitioning between silicate and metal melts; a model for core formation}}
\cvitem{2015}{Myhill, R. et al., Quenchable water-rich aluminous post-stishovite, and implications for water cycling and seismic scatterers in the lower mantle, American Mineralogist, in prep.}
\cvitem{2015}{Novella, D. et al., Melting phase relations in the systems Mg$_2$SiO$_4$-H$_2$O and MgSiO$_3$-H$_2$O at upper mantle conditions, in prep for GCA}
\cvitem{2015}{Dannberg, J. et al., Grain-size dependent convection and seismic observables in the Earth's mantle, in prep.}
\cvitem{2015}{Myhill, R., Excess thermodynamic and elastic properties of mineral and melt solutions: modelling and implications for phase relations and seismic velocities, submitted to Earth and Planetary Science Letters}
\cvitem{2015}{Ishii, T. et al., Generation of pressures over 40 GPa using Kawai-type multi-anvil apparatus with tungsten carbide anvils, submitted to Review of Scientific Instruments}
\cvitem{2015}{Myhill, R. et al., On the P-T-\emph{f}O$_2$ stability of Fe$_4$O$_5$ and Fe$_5$O$_6$-rich phases: a thermodynamic and experimental study, submitted to Contributions to Mineralogy and Petrology.}
\cvitem{2015}{Myhill, R. et al. Hydrous melting and partitioning in and above the mantle transition zone: insights from water-rich MgO-SiO$_2$-H$_2$O experiments, submitted to Geochimica et Cosmochimica Acta.}
\cvitem{2015}{Frost, D. J. and Myhill, R. Chemistry of the lower mantle, accepted, AGU Book chapter}
\cvitem{2015}{Wessel, P. et al. Semiautomatic fracture zone tracking, Geochemistry, Geophysics, Geosystems, \doi{10.1002/2015GC005853}.}
\cvitem{2015}{Pamato, M. G., Myhill, R. et al. Lower mantle water reservoir implied by the extreme stability of a hydrous aluminosilicate, Nature Geoscience, 8:75--79, \doi{10.1038/ngeo2306}.}
\cvitem{2013}{Myhill, R. Slab buckling and its effect on the distributions and focal mechanisms of deep-focus earthquakes, Geophysical Journal International, 192.2:837--853, \doi{10.1093/gji/ggs054}.}
\cvitem{2012}{Myhill, R. and Warren, L. M. Fault plane orientations of deep earthquakes in the Izu-Bonin-Marianas subduction zone, Journal of Geophysical Research, 117:B06307, \doi{10.1029/2011JB009047}.}
\cvitem{2011}{Myhill, R., McKenzie, D. and Priestley, K. The distribution of earthquake multiplets beneath the southwest Pacific, Earth and Planetary Science Letters, 301:87--97, \doi{10.1016/j.epsl.2010.10.023}.}
\cvitem{2011}{Myhill, R. Constraints on evolution of the Mesohellenic Ophiolite from sub-ophiolitic metamorphic rocks, in Wakabayashi, J., and Dilek, Y., eds., M\'elanges: Processes of Formation and Societal Significance: Geological Society of America Special Paper 480:1--20, \doi{10.1130/2011.2480(03)}.}

\subsection{Selected Presentations}
\cvitem{2014}{Volatile-driven melting in the deep mantle. Invited seminar, St Louis University.}
\cvitem{2014}{Ferric iron and its influence on Earth's deep structure. Oral presentation, Geological Society of London.}
\cvitem{2011}{Fault plane orientations of deep earthquakes in the Izu-Bonin-Marianas subduction zone. Oral presentation, American Geophysical Union Fall Meeting.}
\cvitem{2011}{Insights into deep earthquake mechanics. Invited seminar, Bayerisches Geoinstitut.}
\cvitem{2011}{Earthquake distributions and their relationship with structural, chemical and thermobaric heterogeneities in subducting slabs. Poster presentation, European Geophysical Union General Assembly.}
\cvitem{2010}{The search for structure: deep-focus earthquakes. Cambridge Earth Sciences Graduate Seminar.}
\cvitem{2009}{Clustering of deep-focus earthquakes in the southwest Pacific. Poster presentation, American Geophysical Union Fall Meeting.}
\cvitem{2008}{The significance of high temperature low pressure rocks beneath the Mesohellenic Ophiolite. Poster presentation, Greek Institute of Geology and Mineral Exploration (IGME) Field Symposium.}

\clearpage
\section{Skills}
\begin{enumerate}[labelindent=5.2em,leftmargin=*,label=\listitemsymbol]
\item Competent user of \href{http://www.latex-project.org/}{\LaTeX}, Microsoft and Serif Office programs.
\item Competent in the C, C++ and Python programming languages, and basic knowledge of FORTRAN and the OpenGL API.
\item Competent in BASH and HTML scripting.
\item Experience in waveform modelling, including receiver function construction, directivity and focal mechanism analysis and relocation routines.
\item Over 400 hours experience with \href{http:www.metamorph.geo.unimainz.de/thermocalc/}{THERMOCALC}, and a competent user of \href{http:www.perplex.ethz.ch/}{Perple\_X} thermodynamic software. Developer of the \href{https://geodynamics.org/cig/software/burnman/}{burnman} thermoelastic and thermodynamic toolkit.
\item User of the \href{http:www.fenicsproject.org/}{FEniCS} finite element modelling software and the \href{https://geodynamics.org/cig/software/aspect/}{ASPECT} open-source geodynamics software.
\end{enumerate}

\section{Service to the profession}
%\cvitem{2015}{Associate Editor, American Mineralogist} 
\cvitem{2015-}{Software developer for \emph{ASPECT}, open-source software for mantle convection written in C++ (\url{http://geodynamics.org/cig/software/aspect/}).}
\cvitem{2014-}{Software developer for \emph{burnman}, an open-source mineral physics toolkit written in python (\url{http://geodynamics.org/cig/software/burnman/}).}
\cvitem{2014}{Reviewer for \emph{Geochemical Perspectives}.}
\cvitem{2012, 2015}{Session Convener, AGU Fall Meeting.}
\cvitem{2004-}{Scientific outreach related to the engineering and geological sciences (including part of the TimeTruck organising committee, conducting geological outreach at the University of Cambridge).}

\section{Relevant Experience}
\cvitem{06/07-01/08}{Master's Thesis, University of Cambridge. Metamorphic Development beneath the Mesohellenic Ophiolite. In this self-designed project, I used \href{http:www.metamorph.geo.unimainz.de/thermocalc/}{THERMOCALC} to interpret the mineral assemblages in samples collected from the amphibolite-granulite facies metamorphic soles of the Mesohellenic Ophiolite in Northern Greece. I obtained a high First (80\%) for this project. \emph{Advisor: Dr Timothy Holland, University of Cambridge}.}
\cvitem{06/06-01/07}{Bachelor's Thesis, University of Cambridge. Independent mapping project and industrial work experience: Vourinos, Northern Greece. I obtained the top First in the year for this project (80\%). \emph{Advisors: Dr Alan Smith, University of Cambridge and Dr Anne Rassios, IGME}.}
\cvitem{07/05-08/05}{Field geologist, British Geological Survey. Paid appointment for the Tellus Project, part of a national environmental survey completed in 2006. \emph{Advisors: Louise Ander, Sean Quigley, Sophia Passmore (British Geological Survey)}.}
\cvitem{09/08}{Field demonstrator and member of the organising team, IGME Field Symposium: Ophiolites 2008.}
\cvitem{2008--2012}{Supervisor in Stratigraphy, Structural Geology, Hydrosphere and Tectonics, University of Cambridge.}
\cvitem{}{Demonstrator in Structural Geology, Tectonics and Seismology, University of Cambridge.}
\cvitem{}{Demonstrator on Earth Sciences departmental field trips to Ketton, Arran, and Sedbergh.}
%\cvitem{}{Guest field guide, University of K\"{o}ln}{}{Vourinos Ophiolite, Northern Greece}
\cvitem{2005--2012}{Fieldwork experience. 250+ days (as of \today) as demonstrator, field guide, employee, researcher and student (Locations around the UK, Iceland and Greece).}

\section{Other Interests}
\begin{enumerate}[labelindent=5.2em,leftmargin=*,label=\listitemsymbol]
\item First Aid: I have been an active member of St John Ambulance for much of my life. I volunteered for about 400 hours per year during my PhD, primarily as first aider and Duties Coordinator of Cambridge LINKS (the student division of SJA) 2010--2012, and also as Stores Manager in 2011--2012. 
\item Photography: I am a keen amateur photographer. I have particularly enjoyed macro and landscape photography while hiking during my time in Bavaria.
\item Greek culture: I have enjoyed over 6 months since 2006 hiking and conducting fieldwork in Greece, and spent my final year at undergraduate level studying Modern Greek at the Department of Medieval and Modern Languages, University of Cambridge.
\end{enumerate}

\clearpage
\end{document}
